\chapter{Estado da Arte}
As interfaces mais populares para desenvolvimento em sistemas embarcados são majoritariamente desenvolvidas sobre interfaces de programação genéricas e utilizadas para os mais variados fins, e dentre estas, muitas são desenvolvidas utilizando a linguagem de programação JAVA, conhecida por ser uma linguagem de baixo desempenho por ser executado pela JVM (\textit{Java Virtual Machine}).

\abreviatura{JVM}{Java Virtual Machine}

Boa parte destas IDE são de código fechado, com certas restrições para o desenvolvedor, restringindo a liberdade do mesmo no seu fluxo de trabalho, em alguns casos obrigando a utilizar hardwares proprietários para a execução de tarefas voltados a sistemas embarcados, obrigando a empresa ou usuário adquirir soluções desenvolvidas pelos fornecedores. Alem disso, tais sistemas profissionais são conhecidos por terem preços desproporcionais as suas funções comparadas as alternativas de código aberto, podendo custar até milhares de dólares %\itodo{[ACHAR A FONTE DISSO AQUI]}.

Dentro das opções de código aberto de desenvolvimento de sistemas embarcados, pode-se se destacar a Arduino, que disponibiliza tanto o software quanto o hardware. A sua popularidade veio como consequência da sua cultura de desenvolvimento de \textit{Open Source}, onde tanto o hardware, quanto o software distribuídos tem sua documentação por completo aberta, permitindo que inúmeros entusiastas desenvolvam sem a necessidade de restrições, muitos outros acabam colaborando com os próprios projetos da empresa, facilitando desta forma o trabalho da mesma, produzindo versões melhoradas de seus produtos com a ajuda dos próprios usuários e suas contribuições. A IDE desenvolvida pela Arduino e concebida majoritariamente em JAVA.

Outras duas interfaces de código aberto utilizadas para sistemas embarcados são Eclipse e CCS, sendo o segundo baseado no primeiro
derivado pela empresa conhecida como \textit{Texas Instruments}. Baseados na linguagem de programação JAVA, possibilitam uma
integração com plugins relativamente fácil, podendo ser expandidas de acordo com o desenvolvedor e a disponibilidade das mesmas.

\section{IDE Existentes}
Dentro das opções disponíveis, pode-se destacar (Tabela \ref{ides}):
%\itodo{[EXPANDIR DESCRIÇÃO]}
\iffalse
\begin{itemize}
 \item \textbf{Atmel Studio}
 \subitem Limitado aos processadores AVR.
 \subitem Código fechado.
 \subitem Preço: Pago.

 \item \textbf{Arduino IDE}
 \subitem Limitado ao suporte para placas da mesma plataforma com processadores AVR e ARM, alem de algumas personalizadas
 pela comunidade open source, como, por exemplo: ESP-8266
 \subitem Código aberto.
 \subitem Preço: Grátis.

 \item \textbf{ArduIDE}
 \subitem Limitado aos processadores AVR.
 \subitem Código aberto.
 \subitem Preço: Gratuito.

 \item \textbf{PROGRAMINO IDE}
 \subitem Limitado aos processadores AVR.
 \subitem Código fechado.
 \subitem Preço: 30 euros.

 \item \textbf{PlatformIO}
 \subitem Plugin para Atom, que por sua vez é baseado em um browser.
 \subitem Código aberto.
 \subitem Preço: Gratuito.

 \item \textbf{Embrio}
 \subitem Limitado aos processadores AVR.
 \subitem Código fechado.
 \subitem Preço: 59 dólares.

 \item \textbf{Embedded Studio}
 \subitem Para fins educacionais, limitado aos processadores ARM.
 \subitem Código fechado
 \subitem Preço: Gratuito.

 \item \textbf{IAR}
 \subitem Limitado aos processador ARM.
 \subitem Código fechado.
 \subitem Preço: Pago.

 \item \textbf{mikroe IDE}
 \subitem Limitado aos processador ARM da família M3 e M4.
 \subitem Código fechado.
 \subitem Preço: 299 dólares.

 \item \textbf{GNU ARM Eclipse}
 \subitem Limitado aos processadores ARM, plugin para a IDE Eclipse.
 \subitem Código aberto.
 \subitem Preço: Gratuito.

 \item \textbf{CrossWorks}
 \subitem Depende do tipo de licença, limitado aos processador ARM.
 \subitem Código fechado.
 \subitem Preço: \$150 até \$2250.

 \item \textbf{Code Composer Studio}
 \subitem Dependendo do tipo de licença.
 \subitem Código fechado.
 \subitem Preço: Grátis até \$5495.
 
 \item \textbf{Keil}
 \subitem Dependendo do tipo de licença.
 \subitem Código fechado.
 \subitem Preço: Grátis até \$5495.
 
 \item \textbf{Simplicity-studio}
 \subitem Dependendo do tipo de licença.
 \subitem Código fechado.
 \subitem Preço: Pago.

\end{itemize}
\fi

% Please add the following required packages to your document preamble:
% \usepackage{booktabs}
\begin{table}[]
\centering
\caption{IDE existentes}
\label{ides}
\begin{tabular}{@{}llll@{}}
\toprule
Nome                  & Processadores                                          & Código  & Valor              \\ \midrule
ArduIDE               & AVR e ARM                                     & Aberto  & Gratuito           \\
Arduino IDE           & \makecell[l]{AVR, ARM, X86, espressif,\\RISCV entre outros.} & Aberto  & Gratuito           \\
Atmel Studio          & AVR                                           & Fechado & Pago               \\
Code Composer Studio  & \makecell[l]{ARM, DSP, TSM,  MSP\\entre outros.}                      & Fechado & Até 4500 dólares   \\
CrossWorks            & ARM                                                    & Fechado & 150 a 2250 dólares \\
EmbeddedStudio        & ARM                                                    & Fechado & Gratuito           \\
Embrio                & AVR                                                    & Fechado & 59 dólares         \\
GNU ARM Eclipse       & ARM                                                    & Aberto  & Gratuito           \\
IAR                   & \makecell[l]{AVR, ARM, X86,  MSP\\entre outros.}                      & Fechado & Pago  \\
Keil                  & \makecell[l]{ARM, X86, AVR\\entre outros.}                            & Fechado & Até 5495 dólares   \\
mikroe IDE            & ARM                                                    & Fechado & 299 dólares        \\
PlatformIO            & \makecell[l]{AVR, ARM, X86, espressif,\\RISCV entre outros.} & Aberto  & Gratuito           \\
PRAGRAMINO IDE        & AVR                                           & Fechado & 30 euros           \\
Simplicity Studio     & ARM, X86 entre outros.                                 & Fechado & Pago.              \\ \bottomrule
\end{tabular}
\end{table}

\newpage

Maioria das IDEs citadas são de código fechado e pagas, impossibilitando sua modificação. Dentre as gratuitas, a Arduino IDE e PlatformIO tem suporte para múltiplas arquiteturas, contudo, estas IDEs são limitadas no desenvolvimento ou na performance, contendo pouca ou nenhuma opção de \textit{code parsing}, gerenciamento de projeto, entre outros. É conclusivo que dentre as opções citadas nenhuma atende os requisitos do projeto, ressaltando a necessidade deste trabalho para a comunidade de desenvolvedores de sistemas embarcados.


\iffalse
\section{Dificuldades}
Pela constante evolução no mundo de sistemas embarcados, cada empresa acaba criando se próprio padrão de desenvolvimento para os
usuários, tendo o intuito de ajuda-los, acaba prejudicando o trabalho dos desenvolvedores com conhecimento avançado. Tal conhecimento não se trata de um arquitetura de hardware especifica, ou da abstração de software utilizada pelo fabricando para não resultar no uso de software a nível de registrador, mas sim o conhecimento sobre como o processo de gravação é realizado de forma intrínseca na visão de hardware e software integrados.

Boa parte das interfaces de programação para sistemas embarcados se trada de um sistema com vasta quantidade de configurações,
com uma quantidade de abas e janelas que tornar inusual a utilização do gerenciador de projeto. Mesmo o intuito do fabricante ser bom,
isso acaba torna os desenvolvedores escravas da ferramenta, sendo obrigados a conhecer todas as peculiaridades da mesma para a sua
utilização, fazendo com que a curva de aprendizado e desenvolvimento muito engrime.

Outro problema, é a falta de padronização entre os fabricante, nas placas de desenvolvimento, nos gravadores, nas IDEs e nas bibliotecas
básicas utilizadas para desenvolvimento do software de sistema embarcado. Atualmente, um dos padrões mais utilizados para software
de sistemas embarcados, são as nomenclaturas utilizadas pela Arduino e pelo CMSIS, o primeiro por ser o mais utilizado por hobbistas
e o segundo por ser o mais utilizado entre os desenvolvedores que optam pelos processadores derivados da arquitetura ARM.
\fi
\abreviatura{CMSIS}{Cortex Microcontroller Software Interface Standard}
