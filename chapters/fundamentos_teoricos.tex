\chapter{Fundamentos teóricos}

\section{Sistema embarcado}
Antes de adentrar na especificação do sistema embarcado, será apresentado alguns exemplos.

\begin{itemize}
\item \textbf{VANT} - Com crescimento exponencial nos últimos anos, atualmente se encontram com inúmeras apostas em suas utilizações com alto investimento.
\item \textbf{Roteador} - Extremamente populares no acesso da internet sem a utilização de cabos. Muitos contem uma arquitetura computacional completa contendo até sistemas operacionais embarcados.
\item \textbf{GPS} - Muitos já contem sistemas de processamento, podendo realizar filtragem, fusionamento, estimação e permitindo comunicação em inúmeros protocolos.
\item \textbf{Televisões} - Os modelos dos últimos anos tiveram como foco principal a extensão da televisão como ambiente de entretenimento, permitindo conexão com internet, instalação de aplicativos para acesso a serviços online de entretimento e informação.
\item \textbf{Câmeras de vigilância} - Alguns tipos permitem o acesso via internet, utilizando uma eletrônica embarcada com sistemas operacional e periféricos de rede, permitindo o acesso do vídeo via conexão ethernet.
\end{itemize}

\abreviatura{VANT}{\textit{Veículo Aéreo Não Tripulado}}
\abreviatura{GPS}{\textit{\textit{Global Positioning System }}}

Numa visão mais ampla, o sistema embarcado pode ser resumido para qualquer sistema, sendo de alta ou baixa performance energética/computacional, que realiza uma única tarefa. Mesmo podendo afirmar isso hoje, as definições de sistemas embarcados foram modificadas ao passar do tempo, como, por exemplo, o baixo processamento, que se tornou inapropriada ao passar dos anos.

A evolução do hardware, diminuindo o tamanho e preço de aquisição dos processados, permitiu praticamente a qualquer desenvolvedor de sistemas computacionais embarcados um poder de processamento grande o suficiente para rodar um kernel\footnote{Cerne do sistema operacional, tendo como principal função o escalonamento de tarefas e abstração dos periféricos utilizados da maquina, como, por exemplo: placa de rede, sensores, portas de comunicação e etc.} de sistema operacional complexo, como o Linux, em praticamente qualquer aplicação.

Muitas das aplicações de sistemas embarcados são utilizados para sistemas cyber-físicos\footnote{Sistemas computacionais voltados para interação no mundo físico via atuadores e tendo como entrada informações de transdutores, alguns exemplos podem se destacar: Controle de combustível, controladora de VANTs, impressora 3D e etc.}

\subsection{Microcontrolador, microprocessador, microcomputador, SoC}
\lipsum[1-4]

\subsection{Sistema de boot}
\lipsum[1-4]

\subsection{Gravação}
\lipsum[1-4]

\section{KDevelop}
\lipsum[1-4]

\subsection{Plugins}
\lipsum[1-4]

\subsection{Comunidade Open Source}
\lipsum[1-4]

\subsection{KDE}
\lipsum[1-4]

\section{Sistema de Controle de Versão}
\lipsum[1-4]
