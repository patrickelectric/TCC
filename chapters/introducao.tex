\chapter{Introdução}
Nos últimos 40 anos existiram varias predições para o futuro da humanidade, conceitos como IoT(\textit{Internet Of Things})\cite{gates1995estrada} e ambiente ubíquo\cite{weiser1991computer} foram criados, propostas tecnológicas como a \textit{lei de Moore}\footnote{Citação famosa utilizada por Gordon Earle Moore, que tinha como proposta a duplicação do numero de transistores a cada 18 meses} eram ditas como garantidas pelas grandes empresas de tecnologia. Contudo muitos desafios acabaram prejudicando a execução destas predições, como, por exemplo, a barreira de potência\cite{Patterson:2008:COD:1502247} e a limitação de densidade energética\cite{paradiso2005energy}.
%http://www.singularity.com/charts/page70.html

Hoje, o avanço tecnológico permitiu a miniaturização dos processadores com um custo de aquisição de grandezas menores das décadas anteriores\cite{nordhaus2007two}, possibilitando a pequenas empresas e até mesmo pessoas comuns o acesso a sistemas computacionais com alto processamento. Contudo, mesmo	 sendo possível, as ferramentas existentes para permitir o desenvolvimento de aplicações pros processadores, são geralmente pagas (\textit{IAR Embedded Workbench}\footnote{Não existe versão gratuita\cite{buyiar}.}), limitadas (\textit{MDK Microcontroller Development Kit}\footnote{Versão gratuita com limite de código de 32 Kbytes.}) ou com poucos recursos (\textit{Platformio}\footnote{Sem depuração embutida.}).
%http://www2.keil.com/mdk5

A comunidade dos desenvolvedores de software para sistemas embarcados tende a utilizar inúmeras ferramentas para resolver os desafios de projeto, geralmente embutidas em uma IDE (\textit{Integrated Development Environment}), permitindo funções básicas, por exemplo, depuração, comunicação, gravação, entre outros. Algumas IDEs tendem a resolver mais de um problema ao mesmo tempo, resultando num sistema complexo, além disso, possuem a tendência de serem excessivamente assistencialistas (autocompletar e pop-ups sem requisição do usuário), abstraindo os sistemas que executam em segundo plano, resultando num uso maior do processador da maquina de trabalho, dificultando a personalização do fluxo de desenvolvimento e dos sistemas utilizados pelo desenvolvedor.

\iffalse
O intuito deste trabalho é a realização de um sistema para possibilitar aos desenvolvedores a programação de sistemas embarcados, sem a necessidade de utilizar sistemas assistencialistas que possam limitar a evolução do trabalho ou a utilização do produto final concebido.
\fi

O projeto tem como objetivo uma proposta de solução para os problemas descritos anteriormente, não providenciando alguma padronização ou mais uma alternativa entre as já existentes pelo mercado, mas uma IDE que tenha um suporte de alto nível para todos estes que já existem. Tendo como foco de usuário o desenvolvedor de sistemas embarcados, visando um ambiente configurável e mutável a suas necessidades.

Para a concepção, foi utilizado o KDevelop\footnote{IDE desenvolvida pela comunidade internacional de software livre KDE.} para base do sistema definido como sendo proposta do GSoC (\textit{Google Summer Of Code})\footnote{Projeto de incentivo para estudantes no desenvolvimento de código livre organizado pela Google.}. Por ser uma IDE desenvolvida majoritariamente em C++ e permitir o carregamento em tempo de execução de algumas ferramentas (via a utilização de \textit{plugins}), já abrangendo alguns dos requerimentos do sistema que serão discutidos.

\section{Motivação}

Com o avanço tecnológico e a evolução do hardware heterogêneo (sistemas com GPU, CPU e outros periféricos), muitos dos sistemas disponíveis para desenvolvimento não seguem um padrão estabelecido no carregamento de código (gravador, protocolo, \textit{bootloader} e etc.), além disso, as empresas acabam criando alternativas aos padrões já
%https://community.nxp.com/thread/389162
 existentes no mercado ou opções limitadas nas IDEs, como, por exemplo, DebugWire\cite{debugwire}, hardware exclusivo de depuração LPC\cite{nxp}, suporte limitado de depuradores\cite{kiledebug} e etc, dificultando ou restringindo o desenvolvimento do usuário.

Geralmente, as ferramentas utilizadas para o desenvolvimento são de código fechado, distribuídas numa versão binaria, restringindo a modificação do software pelo usuário e personalizações do desenvolvimento. No desenvolvimento do estado da arte, a personalização da ferramenta é essencial para permitir o desenvolvimento, evitando a espera do suporte dentro dos sistemas vendidos, desta forma, muitas empresas acabam aderindo aos software de código aberto pela sua agilidade para adesão a novas tecnologias, podendo destacar Microsoft\cite{microsoftn1}, Intel, Samsung, IBM e Renesas\cite{topskernel}.

\iffalse
\subsubsection{Ferramentas proprietárias}

Pela falta de um padrão definido, muitas empresas resolveram criar seus próprios, fundamentados em seus interesses ou na suas definições de utilização do que deveria ou não conter em suas soluções. Como consequência disso, são criados visões limitadas da realidade, resultante do modo de atuação interno vivida pela empresa, pela sua falta de conhecimento da realidade atual do mercado e suas utilizações.

Como consequência disso, a gama de padrões de protocolos de comunicação, \textit{bootloaders}, ICSPs, compiladores e outros materiais utilizados no desenvolvimento de sistemas embarcados, dificulta a realização do trabalho do desenvolvedor, principalmente se tais periféricos utilizados tem documentação e código fonte fechados, não permitindo seu conhecimento de funcionamento.
\fi
%\itodo{Adicionar footnote the bootloader}
%\subsubsection{Padronização}

\section{Objetivo Geral}


%comment
\iffalse
O objetivo geral deve responder as seguintes perguntas:
1) O que a sua organização deseja realizar com o Projeto?
2) Qual problema em especial se quer solucionar?
3) Que mudanças se quer alcançar?
4) Que diferença o projeto quer fazer?

Deve ser escrito em tempo infinitivo (por exemplo: ampliar, capacitar, entre outros) e redigido com claridade. O objetivo precisa ser alcançável, não pode ser genérico, de forma que o projeto não consiga resolver (ex: terminar com a fome no mundo). Por outro lado deve ser ousado, capaz de sinalizar mudanças mais profundas que poderão ser alcançadas pelo projeto a médio e longo prazo.
\fi

\subsection{Metodologia}
\label{ss:objetivosespecificos}
Para realizar os objetivos, os seguintes passos são listados:
\begin{enumerate}
\item Estudo inicial das ferramentas utilizadas para desenvolvimento de sistemas embarcados.
\item Escolha das ferramentas inicialmente suportadas.
\item Analise inicial sobre KDevelop.
%\footnote{adicionar nota de roda pé sobre o KDevelop, botar isso onde primeiramente acontece.} no irc \itodo{Adicionar abreviatura de irc} \footnote{Adicionar footnote sobre irc}.
\item Estudo para traçado inicial de estrutura e objetivo.
\item \textbf{Inicio do desenvolvimento}.
\item Carregamento do plugin.
\item Selecionar primeiro sistema embarcado para desenvolvimento.
\item Desenvolvimento do instalador de dependências.
\item Desenvolvimento de interface de programação.
\item Realização de testes utilizando o sistema selecionado.
\item Aprimoramento do sistema.
\item Selecionar outro sistema sistema embarcado para realizar o desenvolvimento.
\item Atualização do sistema de instalação.
\item Repetir passo 10 em diante.
\end{enumerate}

\section{Analise do fluxo de trabalho}

A interface de projeto existente no KDevelop, permite que boa parte das especificações de projeto fiquem no sistema de geração automatizada\footnote{Entre eles pode se destacar os mais utilizados são o \textit{Makefile}, \textit{CMakefile} e \textit{WAF}.}, permitindo o desenvolvimento de software sem a integração do projeto com o KDevelop para sua utilização.
\iffalse
, utilizando arquivos intermediários de configuração\footnote{Arquivos que contem informações sobre compilador, estilo de código, execução e etc.}.
\fi
Esta flexibilidade de integração de projetos no KDevelop acrescenta um grau de dificuldade na realização do plugin, pois, as informações necessárias no carregamento do binário para o sistema embarcado necessita de sua localização e da porta de comunicação para o hardware em desenvolvimento. %\itodo{ADICIONAR MAIS INFORMAÇÕES DE FLUXO}

\section{Estrutura do Trabalho}

\begin{enumerate}
\item \textbf{Introdução} - Tema e contextualização do que será abordado no trabalho.
\item \textbf{Fundamentos teóricos} - Revisão dos assuntos abordados e necessários, fundamentais para realização do projeto.
\item \textbf{Estado da arte} - Comparação dos sistemas existentes que abordam o assunto e tentam aplicar uma proposta de solução.
\item \textbf{Projeto da ferramenta} - Rascunhos iniciais de interface e arquitetura de software.
\item \textbf{Desenvolvimento} - Relato do trabalho realizado, contendo informações do projeto e seu andamento.
\item \textbf{Experimento e resultados} - Exposição dos resultados.
\item \textbf{Conclusão} - Conclusões finais.
\end{enumerate}
%\itodo{Expandir de forma mais detalhada}