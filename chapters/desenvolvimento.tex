\chapter{Desenvolvimento}

Diferente de muitos outras interfaces de desenvolvimento, o KDEVElop mermiti o usuario incorporar qualquer projeto na IDE,
des de que siga alguns padrões já bem estabelecidos na cultura open source de desenvolvimento de software realizado e C++.
Pois, boa parte das configurações que são buscadas na IDE para a configuração do projeto, vem dos gerenciadores de compilação,
como antes citados, CMAKE e MAKE. Com isso, não é necessario lidar com as burocracias de incorporaçãode projeto, ou até mesmo
o vinculo do projeto a uma certa IDE, impossibilitando o desenvolvimento do mesmo em outros ambiente e fluxos de trabalho.
Os projetos que são realizados com as ferramentas de gerenciamento de compilação, ajuda muito pela sua generalização e
distribuição, por serem projetos de códigos abertos extremamente populares e comuns em projetos de linguagens não interpretadas.

\section{Exposição do tema ou matéria}

Sendo a área de desenvolvimento de sistemas embarcados potencializado nos utlimos anos pela miniaturuzalão do hardware, permitindo
a cada dia um paço mais próximo a um abiente ubiquo com contribuição das bases de desenvolvimento das tecnologias de IoT. Contudo,
como dito anteriormente, os sistemas utilizados atualmente não permitem um fluxo de trabalho agradavel para desevolvedores, muito menos
para empresas de pequeno, médio e até mesmo de grande porte, que se limita as seguintes categorias:

\begin{itemize}
 \item \textbf{Ferramentas proprietárias}: Ferramentas de desenvolvimento para resolver problemas criados pela própria empresa que os fornece,
 obrigando a utilização de periféricos da mesma distribuidora ou parceira, fazendo limitações de liberdade de desenvolvimento via
 software. Outras permitem o desenvolvimento limitado a licenças de valores ezorbitantes, fazendo com que pequenas empresas que utilizam
 um baixo capital para se manterem utilizem licenças de baixo custo que limita a utilização do software de maneira praticamente
 ilega, xomo, por exemplo: Pagar uma licença mais cara para utilizar opções de compilação otimizada para uma especifica arquitetura
 como (-O2 e -OS do GCC) ou até mesmo opções para utilizar o algumas partes do processador (-mfloat-abi e -mfpu dos processadores ARM,
 permitindo execução de calculos matriccial ou com ponto flutuante em hardware).
 \item \textbf{Ferramentas gráficas de código aberto}: Utilizadas por uma parte dos desenvolvedores
 \item \textbf{Ferramentas com \textit{cli}}: Geralmente utilizadas pelas interfaces gráficas como \textit{back-end} para realizar suas
 operações em relação ao sistema embarcado, as ferramentas utilizadas seriam: OpenOCD, Avrdude, esptool, dfu-utils entre outros.
\end{itemize}

\abreviatura{cli}{\textit{Command-line interface}}


\subsection{Ferramentas Utilizadas}
As seguintes ferramentas foram utilizadas para a realização do desenvolvimento do projeto.
\begin{itemize}
 \item \textbf{KDevelop}: Editor de código para programar o plugin e realizar os testes do mesmo nos sistemas embarcados utilizados,
 também utilizado para a realização de debug.
 \item \textbf{Placas de desenvolvimento}: Arduino mini, nano, mega, due, uno para realizar testes no suporte realizado para Arduino
 utilizando avrdude. Stellaris LM4F232 para testes utilizando o suporte feito do OpenOCD.
 \item \textbf{GIT}: Utilizado para realização do controle de versão e documentação do  histórico de desenvolvimento.
 \item \textbf{Astyle}: Ferramenta utilizada para checar e corrigir estilo do código.
 \item \textbf{Codespell}: Script para correção ortográfica da documentação e do código.
 \item \textbf{Doxygen}: Programa assistencialista para geração de documentação.
 \item \textbf{Arch Linux}: Sistema operacional utilizado para realização do desenvolvimento e execução de programas.
 \item \textbf{Valgrind}: Utilizado para realizar a analise de execução do código.
 \item \textbf{KCachegrind}: Ferramenta para visualizar o uso de memória do programa, util para encontrar \textit{memory leaks}.
 \item \textbf{QT Assistant}: Programa assistencialista para visualização a documentação das bibliotecas fornecidades pela QT.
 \item \textbf{IRC}: Para retirar duvidas e entrar em contato com outros desenvolvedores do KDevelop.
\end{itemize}

Com o objetivo do trabalho de seguir a filosofia GNU \cite{filosogia}, todos as ferramentas utilizadas para o desenvolvimento do plugin
são de código aberto e totalmente gratuitas para qualquer um que queira replicar, modificar e contribuir com o projeto.

\subsection{Protótipo de funcionamento}

\subsection{Integração com KDevelop}

\section{Integração com fluxo de trabalho}
Além do desenvolvimento do software, permitindo seu uso como projetado inicialmente, foi necessario reservar uma parte do periodo
do projeto para modificações de interface gráfica e de organização de layout, para permitir o uso relativamente simples para usuarios
não tão avançados, e ao mesmo tempo, permitir uma flexibilidade aos usuarios com conhecimento profundo do funcionamento do sistema
utilizado no background do projeto\footnote{OpenOCD, avrdude entre outras futuros sistemas opensources que devem serincorporados no
plugin}, modificando e personalizando as opções fornecidades pelo ambiente gráfico sem restringir o alto nivel de abstração que essas
ferramentas já permitem e prejudicar a liberdade de configurações disponiveis para o usuario.
