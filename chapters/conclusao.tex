\chapter{Resultados experimentais}
Após a finalização duma versão usável para qualquer desenvolvedor com conhecimento mediano para avançado sobre o desenvolvimento de
sistemas embarcados, teve o inicio das execuções experimentais para a validação do sistema proposto pelo trabalho.

Para sua realização,


\chapter{Conclusão}
Dentro do periodo de desenvolvimento, o plugin alcançou todas as espectativas traçadas para para o periodo de desenvolvimento,
permitindo a gravação dos sistemas embarcados utilizando o KDevelop, mesmo necessitando algumas modificações para ficar
totalmente integrado ao fluxo de trabalho, o plugin se encontro numa forma usavel.

A integração com o KDevelop foi feita com sucesso, a incorporação de projetos funciona, necessitando que o usuario configure algumas
coisas

\section{Contribuições}

\section{Futuros Trabalhos}


\subsection{Instalador}
Mesmo o instalador sendo completamente funcional, identificando versões instaladas no sistema operacional que está o executando e
permitindo ao usuario realizar a instação automatica das dependencias com a ajuda da interface gráfica disponivel pelo plugin,
o mesmo não é de muita ajuda para os usuarios avançados, e até mesmo para os inciantes, pois ambos já teriam conhecimento de como
instalar e atualizar a ferramenta, pois, como dissertado antes sobre a cultura do KDE de ser \textit{rolling release}, a
instalação e atualizações de tais dependencias seriam gerenciadas pelo usuario sem problemas, fazendo desta forma que o instalador
desenvolvido par ao projeto seja removido em futuras versões.

\subsection{Melhora na integração ao sistemas de compilação}
A integração do plugin desenvolvido com os demais plugins do KDevelop necessita ser melhor desenvolvida e aprimorada, permitindo desta
forma a integração com os gerenciadores de montagem de software, permitindo com que o kdev-embedded detecte a localização do arquivo
binario para realizar o processo de programação do sistema embarcado e até mesmo utilizar tal integração para permitir ao plugin
detectar o processador do sistema embarcado em desenvolvimento que está sendo utilizado para permitir o envio do código binario para
o hardware.

\subsection{Melhora no identificador de hardware}
Atualmente o plugin utiliza uma biblioteca desenvolvida pela comunidade do KDE conhecida como Solid, onde permite uma boa integração
com o hardwre do computador onde o mesmo está sendo executado. Contudo o mesmo tem problemas para identificação

