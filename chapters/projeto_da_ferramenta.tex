%http://www.linux-magazine.com/Online/Blogs/Paw-Prints-Writings-of-the-maddog/Brazil-Free-and-Open-Source-Culture-Is-Economics-Not-Politics
%https://books.google.com.br/books?id=pxAVCgAAQBAJ&pg=PA166&lpg=PA166&dq=Brazil:+Free+and+Open+Source+Culture+Is+Economics,+Not+Politics&source=bl&ots=k5M51pGklX&sig=tCPn4l7OnL7Xzne9cKYowUbYTvo&hl=pt-BR&sa=X&ved=0ahUKEwihkIec0-HPAhUGjpAKHdv6AcwQ6AEIOzAE#v=onepage&q=Brazil%3A%20Free%20and%20Open%20Source%20Culture%20Is%20Economics%2C%20Not%20Politics&f=false
%http://pear.accc.uic.edu/ojs/index.php/fm/article/view/904/813
%https://books.google.com.br/books?id=fEP2smsHVCwC&pg=PA14&lpg=PA14&dq=Brazil:+Free+and+Open+Source+Culture+Is+Economics,+Not+Politics&source=bl&ots=FsxRuW6joV&sig=l_PNwCwjTLl-XBATmeSIVO70Fcs&hl=pt-BR&sa=X&ved=0ahUKEwihkIec0-HPAhUGjpAKHdv6AcwQ6AEISzAG#v=onepage&q=Brazil%3A%20Free%20and%20Open%20Source%20Culture%20Is%20Economics%2C%20Not%20Politics&f=false
\chapter{Projeto da ferramenta}
O rascunho inicial da proposta, tem como intuito facilitar a visualização da ferramenta, facilitando o planejamento do plugin.

%\itodo{Descrição da janela de instalação}
%\itodo{Adicionar mockup da janela de instalação}

%\itodo{Descrição da janela de configuração de projeto}
%\itodo{Adicionar mockup da janela de configuração}

\section{Projeto do desenvolvimento}

KDE é conhecido por utilizar as ferramentas e bibliotecas disponibilizadas pela \textit{QT Project}\footnote{Empresa desenvolvedora  de software, famosa pelas suas bibliotecas e \textit{IDEs} de desenvolvimento} para a realização do desenvolvimento dos seus projetos. O conhecimento inicial de como utilizar tais ferramentas é de grande valia para o projeto do desenvolvimento da proposta, principalmente pelo fato de se conhecer as possibilidades.

\subsection{Especificações de requisito do sistema}

O usuário deve ser capaz de utilizar o plugin para realizar a programação de sistemas embarcados, com a ajuda de um \textit{bootloader}, ou utilizando programadores direto no hardware. Além de ser possível programar, dependendo do caso, o desenvolvedor seria capaz de utilizar o plugin para a realização da depuração do projeto dentro do próprio sistema em desenvolvimento, caso o sistema utilizado por suportar tal comportamento.

Os primeiros sistemas que serão suportados pelo plugin como prova de conceito, serão o microcontrolador Arduino utilizando avrdude e outros sistemas embarcados em geral adicionando o suporte para OpenOCD. A escolha do Arduino se deu ao fato pela sua fácil aquisição, baixo preço e popularidade dentre a comunidade civil como um dos microcontroladores mais populares, desta forma, seria um bom primeiro passo para a realização da estrutura de software a ser desenvolvida para o plugin, facilitando desta maneira futuros suportes a serem adicionados, como citado anteriormente, o OpenOCD, a partir da da base criada e testada do Arduino.

Resumindo, deverá ser possível:
\begin{itemize}
\item Programar o código fonte na interface.	
\item Criação de projeto.
\item Compilação do código fonte.
\item Carregamento de código.
\item Depuração do sistema embarcado na interface gráfica.
\item Modificar o sistema e personalizar.
\end{itemize}

\abreviatura{OpenOCD}{\textit{Open On-Chip Debug}}
%Como objetivo final do projeto, o usuário deve ser capaz de utilizar o
