%http://www.linux-magazine.com/Online/Blogs/Paw-Prints-Writings-of-the-maddog/Brazil-Free-and-Open-Source-Culture-Is-Economics-Not-Politics
%https://books.google.com.br/books?id=pxAVCgAAQBAJ&pg=PA166&lpg=PA166&dq=Brazil:+Free+and+Open+Source+Culture+Is+Economics,+Not+Politics&source=bl&ots=k5M51pGklX&sig=tCPn4l7OnL7Xzne9cKYowUbYTvo&hl=pt-BR&sa=X&ved=0ahUKEwihkIec0-HPAhUGjpAKHdv6AcwQ6AEIOzAE#v=onepage&q=Brazil%3A%20Free%20and%20Open%20Source%20Culture%20Is%20Economics%2C%20Not%20Politics&f=false
%http://pear.accc.uic.edu/ojs/index.php/fm/article/view/904/813
%https://books.google.com.br/books?id=fEP2smsHVCwC&pg=PA14&lpg=PA14&dq=Brazil:+Free+and+Open+Source+Culture+Is+Economics,+Not+Politics&source=bl&ots=FsxRuW6joV&sig=l_PNwCwjTLl-XBATmeSIVO70Fcs&hl=pt-BR&sa=X&ved=0ahUKEwihkIec0-HPAhUGjpAKHdv6AcwQ6AEISzAG#v=onepage&q=Brazil%3A%20Free%20and%20Open%20Source%20Culture%20Is%20Economics%2C%20Not%20Politics&f=false
\chapter{Projeto da ferramenta}

KDE é conhecido por utilizar as ferramentas e bibliotecas disponibilizadas pela \textit{QT Project}\footnote{Empresa desenvolvedora  de software, famosa pelas suas bibliotecas e \textit{IDEs} de desenvolvimento} na realização do desenvolvimento dos seus projetos. O conhecimento inicial de como utilizar tais ferramentas é de grande valia para o projeto no desenvolvimento da proposta.


\section{Requisitos funcionais}

O usuário deve ser capaz de utilizar o plugin para realizar a programa-\\ção de sistemas embarcados, com a ajuda de um \textit{bootloader}, ou utilizando programadores direto no hardware. Além de ser possível programar, dependendo do caso, o desenvolvedor seria capaz de utilizar o plugin para a realização da depuração do projeto dentro do próprio sistema em desenvolvimento, caso o sistema utilizado por suportar tal comportamento.

A escolha do Arduino se deu pelo fato de sua fácil aquisição, baixo preço e alta popularidade dentre a comunidade civil como um dos microcontroladores mais populares, desta forma, é um bom primeiro passo para a realização da estrutura de software a ser desenvolvida para o plugin.

O sistema deve ter as seguintes funcionalidades para exercer um bom funcionamento para o usuário.
\begin{itemize}
\item O sistema deve permitir ao usuário:
	\subitem A criação de novos projetos e disponibilizar um modelo.
	\subitem Integração de projeto do usuário.
    \subitem Configuração das ferramentas utilizadas.
	\subitem Executar a compilação do projeto.
	\subitem Executar a instalação de ferramentas.
	\subitem Carregar o binário para o sistema embarcado.
\item Os projetos devem ser salvos assim como as configurações.
\item Executar o carregamento utilizando as ferramentas selecionadas.
\item Avisar sobre erros e problemas durante a ocorrência de alguma etapa para o usuário.	
\end{itemize}

\section{Requisitos não funcionais}
Algumas propriedades e restrições do sistema são vitais para seu funcionamento.
\begin{itemize}
\item O sistema deve executar sem acarretar em uma falha durante a execução.
\item A instalação de ferramentas devem ser feitas sem a permissão de administrador.
\item O programa deve executar sem vazamento de memória.
\item Durante a instalação das dependências, deve ser utilizado nomes randômicos durante o download por segurança, evitando a manipulação do arquivo por terceiros.
\end{itemize}

\section{Plataformas suportadas}

Neste projeto, como versão inicial, será suportado as plataformas mais populares da Arduino, validando o suporte para outras placas com processadores AVR da plataforma do Arduino, e os sistemas que utilizam \textit{OpenOCD}\footnote{Utilizando majoritariamente em placas com processadores ARM (Arduino DUE, STM32, entre outros).}, outros suportes podem vir a ser adicionados em futuras versões.


%Em futuras versões mais  ferramentas deverão ser adicionados,
%, serão os primeiros a serem suportados pelo plugin como prova de conceito por apresentarem uma grande quantidade de usuários.
%Como objetivo final do projeto, o usuário deve ser capaz de utilizar o
